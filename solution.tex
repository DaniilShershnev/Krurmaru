\documentclass[12pt,a4paper]{article}
\usepackage[utf8]{inputenc}
\usepackage[russian]{babel}
\usepackage{amsmath}
\usepackage{amsfonts}
\usepackage{amssymb}
\usepackage{graphicx}
\usepackage{geometry}
\usepackage{tikz}
\geometry{left=2cm,right=2cm,top=2cm,bottom=2cm}

\title{Контрольная работа №1 \\ Уравнения математической физики}
\author{}
\date{}

\begin{document}

\maketitle

\section*{Задача 1. Классификация уравнений второго порядка}

\textbf{Условие:} Решить уравнение, классифицировать его, привести к каноническому виду и найти общее решение:
\begin{equation}
(x-y)u_{xy} - u_x + u_y = 0
\end{equation}

\subsection*{Решение:}

\textbf{Шаг 1. Классификация уравнения.}

Запишем уравнение в стандартной форме:
$$Au_{xx} + 2Bu_{xy} + Cu_{yy} + Du_x + Eu_y + Fu = G$$

В нашем случае:
\begin{itemize}
    \item $A = 0$ (нет члена $u_{xx}$)
    \item $2B = x-y$, то есть $B = \frac{x-y}{2}$
    \item $C = 0$ (нет члена $u_{yy}$)
    \item $D = -1$
    \item $E = 1$
    \item $F = 0$
    \item $G = 0$
\end{itemize}

Вычислим дискриминант:
$$\Delta = B^2 - AC = \left(\frac{x-y}{2}\right)^2 - 0 \cdot 0 = \frac{(x-y)^2}{4} > 0 \text{ при } x \neq y$$

\textbf{Вывод:} Уравнение \textbf{гиперболического типа} при $x \neq y$.

\textbf{Шаг 2. Нахождение характеристик.}

Для уравнения с $A = 0$ и $C = 0$ характеристики находятся из уравнения:
$$A(dy)^2 - 2B\,dx\,dy + C(dx)^2 = 0$$
$$-(x-y)\,dx\,dy = 0$$

Это уравнение выполняется при $dx = 0$ или $dy = 0$, что дает два семейства характеристик:
\begin{align}
\xi &= x = \text{const} \\
\eta &= y = \text{const}
\end{align}

\textbf{Вывод:} Уравнение уже записано в канонических переменных $(x, y)$.

\textbf{Шаг 3. Нахождение общего решения.}

Будем искать решения различного вида.

\textbf{а) Решения вида $u = f(x) + g(y)$:}

Если $u = f(x) + g(y)$, то:
$$u_x = f'(x), \quad u_y = g'(y), \quad u_{xy} = 0$$

Подставляя в уравнение:
$$(x-y) \cdot 0 - f'(x) + g'(y) = 0$$
$$g'(y) = f'(x)$$

Так как левая часть зависит только от $y$, а правая только от $x$, обе части равны константе $C$:
$$f'(x) = C, \quad g'(y) = C$$
$$f(x) = Cx + C_1, \quad g(y) = Cy + C_2$$

Получаем первое семейство решений:
$$u_1 = C(x+y) + D$$

\textbf{Проверка:}
$$u_x = C, \quad u_y = C, \quad u_{xy} = 0$$
$$(x-y) \cdot 0 - C + C = 0 \quad \checkmark$$

\textbf{б) Решения вида $u = F(xy)$:}

Пусть $u = F(s)$, где $s = xy$. Тогда:
$$u_x = F'(s) \cdot y, \quad u_y = F'(s) \cdot x, \quad u_{xy} = F''(s) \cdot xy + F'(s)$$

Подставляя в уравнение:
$$(x-y)[F''(s) \cdot xy + F'(s)] - F'(s) \cdot y + F'(s) \cdot x = 0$$
$$(x-y) \cdot xy \cdot F''(s) + (x-y)F'(s) - yF'(s) + xF'(s) = 0$$
$$(x-y) \cdot xy \cdot F''(s) + 2(x-y)F'(s) = 0$$
$$(x-y)[xy \cdot F''(s) + 2F'(s)] = 0$$

При $x \neq y$:
$$s \cdot F''(s) + 2F'(s) = 0$$
$$\frac{F''(s)}{F'(s)} = -\frac{2}{s}$$

Интегрируя:
$$\ln|F'(s)| = -2\ln|s| + \text{const}$$
$$F'(s) = \frac{A}{s^2}$$
$$F(s) = -\frac{A}{s} + B = -\frac{A}{xy} + B$$

Получаем второе семейство решений:
$$u_2 = -\frac{A}{xy} + B$$

\textbf{Проверка:}
$$u_x = \frac{A}{x^2 y}, \quad u_y = \frac{A}{xy^2}, \quad u_{xy} = -\frac{A}{x^2 y^2}$$
$$(x-y) \cdot \left(-\frac{A}{x^2 y^2}\right) - \frac{A}{x^2 y} + \frac{A}{xy^2} = -\frac{A(x-y)}{x^2 y^2} - \frac{Ay}{x^2 y^2} + \frac{Ax}{x^2 y^2}$$
$$= \frac{-Ax + Ay - Ay + Ax}{x^2 y^2} = 0 \quad \checkmark$$

\textbf{в) Решения вида $u = G(x/y)$:}

Пусть $u = G(t)$, где $t = x/y$. Тогда:
$$u_x = G'(t) \cdot \frac{1}{y}, \quad u_y = G'(t) \cdot \frac{-x}{y^2}$$
$$u_{xy} = -\frac{x}{y^3}G''(t) - \frac{1}{y^2}G'(t)$$

Подставляя и упрощая (умножая на $y^3$):
$$-x(x-y)G''(t) - 2xyG'(t) = 0$$

При $x \neq 0$:
$$(x-y)G''(t) + 2yG'(t) = 0$$

Учитывая, что $x = ty$:
$$y(t-1)G''(t) + 2yG'(t) = 0$$
$$(t-1)G''(t) + 2G'(t) = 0$$
$$\frac{G''(t)}{G'(t)} = -\frac{2}{t-1}$$

Интегрируя:
$$G'(t) = \frac{A}{(t-1)^2}$$
$$G(t) = -\frac{A}{t-1} + B$$

Следовательно:
$$u = -\frac{A}{x/y - 1} + B = -\frac{Ay}{x-y} + B$$

Получаем третье семейство решений:
$$u_3 = \frac{Cy}{x-y} + D$$

\textbf{Проверка:}
$$u_x = \frac{Cy}{(x-y)^2}, \quad u_y = -\frac{Cx}{(x-y)^2}$$
$$u_{xy} = \frac{C}{(x-y)^2} + \frac{2Cy}{(x-y)^3}$$
$$(x-y)\left[\frac{C}{(x-y)^2} + \frac{2Cy}{(x-y)^3}\right] - \frac{Cy}{(x-y)^2} - \frac{Cx}{(x-y)^2}$$
$$= \frac{C(x-y) + 2Cy - Cy - Cx}{(x-y)^2} = \frac{Cx - Cy + 2Cy - Cy - Cx}{(x-y)^2} = 0 \quad \checkmark$$

\textbf{г) Аналогично можно показать, что} $u_4 = \frac{Kx}{x-y}$ также является решением.

\subsection*{Общее решение:}

Поскольку уравнение линейное и однородное, общее решение представляет собой линейную комбинацию найденных решений:

\begin{equation}
\boxed{u(x,y) = A(x+y) + \frac{B}{xy} + \frac{Cx + Dy}{x-y} + E}
\end{equation}

где $A, B, C, D, E$ --- произвольные константы.

Или в эквивалентной форме:
\begin{equation}
\boxed{u(x,y) = C_1(x+y) + \frac{C_2}{xy} + \frac{C_3 x}{x-y} + \frac{C_4 y}{x-y} + C_5}
\end{equation}

\newpage

\section*{Задача 2. Задача Коши для волнового уравнения с источником}

\textbf{Условие:} Решить задачу Коши:
\begin{equation}
\begin{cases}
u_{tt} = 25u_{xx} + \sin x \\
u|_{t=0} = 1 \\
u_t|_{t=0} = 1
\end{cases}
\end{equation}

\subsection*{Решение:}

Уравнение $u_{tt} = 25u_{xx} + \sin x$ является неоднородным волновым уравнением со скоростью распространения $c = 5$.

\textbf{Шаг 1. Решение однородного уравнения.}

Рассмотрим сначала однородное уравнение:
$$u_{tt} = 25u_{xx}$$

с начальными условиями:
$$u|_{t=0} = \varphi(x), \quad u_t|_{t=0} = \psi(x)$$

По формуле Даламбера решение имеет вид:
$$u_0(x,t) = \frac{\varphi(x-ct) + \varphi(x+ct)}{2} + \frac{1}{2c}\int_{x-ct}^{x+ct} \psi(s)\,ds$$

\textbf{Шаг 2. Учет неоднородности через интеграл Дюамеля.}

Для неоднородного уравнения $u_{tt} = c^2 u_{xx} + f(x,t)$ с нулевыми начальными условиями решение дается интегралом Дюамеля:
$$u_f(x,t) = \int_0^t \left[\frac{1}{2c}\int_{x-c(t-\tau)}^{x+c(t-\tau)} f(s,\tau)\,ds\right] d\tau$$

В нашем случае $f(x,t) = \sin x$ (не зависит от $t$), поэтому:
$$u_f(x,t) = \int_0^t \left[\frac{1}{2c}\int_{x-c(t-\tau)}^{x+c(t-\tau)} \sin s\,ds\right] d\tau$$

\textbf{Шаг 3. Вычисление интеграла от источника.}

Сначала вычислим внутренний интеграл:
$$\int_{x-c(t-\tau)}^{x+c(t-\tau)} \sin s\,ds = [-\cos s]_{x-c(t-\tau)}^{x+c(t-\tau)}$$
$$= -\cos[x+c(t-\tau)] + \cos[x-c(t-\tau)]$$

Используя формулу $\cos(\alpha) - \cos(\beta) = -2\sin\left(\frac{\alpha+\beta}{2}\right)\sin\left(\frac{\alpha-\beta}{2}\right)$:

$$\cos[x-c(t-\tau)] - \cos[x+c(t-\tau)] = 2\sin(x)\sin[c(t-\tau)]$$

Поэтому:
$$u_f(x,t) = \int_0^t \frac{1}{2c} \cdot 2\sin(x)\sin[c(t-\tau)]\,d\tau = \frac{\sin x}{c}\int_0^t \sin[c(t-\tau)]\,d\tau$$

Делаем замену $w = c(t-\tau)$, $dw = -c\,d\tau$:
$$u_f(x,t) = \frac{\sin x}{c} \cdot \frac{1}{c}\int_0^{ct} \sin w\,dw = \frac{\sin x}{c^2}[-\cos w]_0^{ct}$$
$$= \frac{\sin x}{c^2}(1 - \cos(ct))$$

При $c = 5$:
$$u_f(x,t) = \frac{\sin x}{25}(1 - \cos(5t))$$

\textbf{Шаг 4. Решение с начальными условиями.}

Теперь применим формулу Даламбера для однородного уравнения с нашими начальными условиями:
$$\varphi(x) = 1, \quad \psi(x) = 1$$

$$u_0(x,t) = \frac{1 + 1}{2} + \frac{1}{10}\int_{x-5t}^{x+5t} 1\,ds = 1 + \frac{1}{10} \cdot 10t = 1 + t$$

\textbf{Шаг 5. Полное решение.}

Полное решение задачи Коши:
$$u(x,t) = u_0(x,t) + u_f(x,t)$$

\begin{equation}
\boxed{u(x,t) = 1 + t + \frac{\sin x}{25}(1 - \cos(5t))}
\end{equation}

\textbf{Проверка решения:}

Проверим начальные условия:
$$u(x,0) = 1 + 0 + \frac{\sin x}{25}(1 - \cos 0) = 1 + 0 = 1 \quad \checkmark$$

$$u_t = 1 + \frac{\sin x}{25} \cdot 5\sin(5t) = 1 + \frac{\sin x}{5}\sin(5t)$$
$$u_t(x,0) = 1 + \frac{\sin x}{5} \cdot 0 = 1 \quad \checkmark$$

Проверим уравнение:
$$u_{tt} = \frac{\sin x}{25} \cdot 25 \cdot (-\cos(5t)) = -\sin x \cos(5t)$$
$$u_{xx} = \frac{-\sin x}{25}(1 - \cos(5t))$$
$$25u_{xx} = -\sin x(1 - \cos(5t)) = -\sin x + \sin x \cos(5t)$$
$$u_{tt} - 25u_{xx} = -\sin x \cos(5t) + \sin x - \sin x \cos(5t) = \sin x \quad \checkmark$$

\newpage

\section*{Задача 3. Построение профиля струны}

\textbf{Условие:} Построить профиль струны для волнового уравнения:
\begin{equation}
\begin{cases}
u_{tt} = c^2 u_{xx} \\
u|_{t=0} = \varphi(x) \text{ (по рис. 9)} \\
u_t|_{t=0} = 0
\end{cases}
\end{equation}

где начальная форма струны $\varphi(x)$ задана графиком:

\begin{center}
\begin{tikzpicture}[scale=1.2]
    \draw[->] (-3.5,0) -- (3.5,0) node[right] {$x$};
    \draw[->] (0,0) -- (0,3.5) node[above] {};

    % Треугольный профиль
    \draw[thick] (-2.5,0) node[below] {$-l$} -- (0,3) node[above] {$h$} -- (2.5,0) node[below] {$l$};

    % Пунктирные линии
    \draw[dashed] (-0.625,0) node[below] {$-l/4$} -- (-0.625,1.5);
    \draw[dashed] (0.625,0) node[below] {$l/4$} -- (0.625,1.5);
    \draw[dashed] (-0.625,1.5) -- (0.625,1.5) node[right] {$h/2$};

    % Точки
    \fill (0,0) circle (1.5pt) node[below] {$0$};
    \fill (0,3) circle (1.5pt);
    \fill (-2.5,0) circle (1.5pt);
    \fill (2.5,0) circle (1.5pt);
\end{tikzpicture}
\end{center}

\subsection*{Решение:}

\textbf{Шаг 1. Запись начальной функции.}

Из графика видно, что начальная форма струны:
$$\varphi(x) = \begin{cases}
h\left(1 + \frac{x}{l}\right), & -l \leq x \leq 0 \\
h\left(1 - \frac{x}{l}\right), & 0 \leq x \leq l \\
0, & |x| > l
\end{cases}$$

Это можно записать компактно:
$$\varphi(x) = \begin{cases}
h\left(1 - \frac{|x|}{l}\right), & |x| \leq l \\
0, & |x| > l
\end{cases}$$

\textbf{Шаг 2. Применение формулы Даламбера.}

Для задачи Коши с $u_t|_{t=0} = 0$ формула Даламбера имеет вид:
$$u(x,t) = \frac{\varphi(x-ct) + \varphi(x+ct)}{2}$$

Эта формула показывает, что решение представляет собой суперпозицию двух волн:
\begin{itemize}
    \item Волна $\frac{1}{2}\varphi(x-ct)$ --- движется вправо со скоростью $c$
    \item Волна $\frac{1}{2}\varphi(x+ct)$ --- движется влево со скоростью $c$
\end{itemize}

Каждая из этих волн имеет амплитуду $h/2$ (половину от исходной).

\textbf{Шаг 3. Построение профиля в различные моменты времени.}

\textbf{При $t = 0$:}
$$u(x,0) = \frac{\varphi(x) + \varphi(x)}{2} = \varphi(x)$$
--- исходный треугольный профиль с вершиной в точке $(0, h)$.

\textbf{При $0 < t < \frac{l}{c}$:}

Правая полуволна (с центром в $ct$):
$$\varphi_R(x,t) = \frac{1}{2}\varphi(x-ct) = \begin{cases}
\frac{h}{2}\left(1 - \frac{|x-ct|}{l}\right), & |x-ct| \leq l \\
0, & |x-ct| > l
\end{cases}$$

Левая полуволна (с центром в $-ct$):
$$\varphi_L(x,t) = \frac{1}{2}\varphi(x+ct) = \begin{cases}
\frac{h}{2}\left(1 - \frac{|x+ct|}{l}\right), & |x+ct| \leq l \\
0, & |x+ct| > l
\end{cases}$$

Правая полуволна занимает область $[ct-l, ct+l]$, левая --- $[-ct-l, -ct+l]$.

\textbf{При $t = \frac{l}{c}$:}

Две полуволны только касаются в точке $x = 0$. Каждая имеет высоту $h/2$.

\textbf{При $t > \frac{l}{c}$:}

Волны полностью расходятся. Правая волна движется вправо, левая --- влево. Между ними образуется плоская область, где $u = 0$.

\subsection*{Графическое представление:}

\begin{center}
\begin{tikzpicture}[scale=0.9]
    % t = 0
    \begin{scope}[shift={(0,0)}]
        \draw[->] (-3.5,0) -- (3.5,0) node[right] {$x$};
        \draw[->] (0,0) -- (0,3.5);
        \draw[thick,blue] (-2.5,0) -- (0,3) -- (2.5,0);
        \node at (0,4) {$t = 0$};
        \node at (0,3.3) {$h$};
        \node at (-2.5,-0.3) {$-l$};
        \node at (2.5,-0.3) {$l$};
    \end{scope}

    % t = l/(4c)
    \begin{scope}[shift={(8,0)}]
        \draw[->] (-3.5,0) -- (3.5,0) node[right] {$x$};
        \draw[->] (0,0) -- (0,3.5);
        \draw[thick,red] (-3.125,0) -- (-0.625,1.5) -- (0.625,0);
        \draw[thick,red] (-0.625,0) -- (0.625,1.5) -- (3.125,0);
        \node at (0,4) {$t = \frac{l}{4c}$};
        \node at (1.2,1.8) {$h/2$};
        \node at (-1.2,1.8) {$h/2$};
    \end{scope}
\end{tikzpicture}

\vspace{1cm}

\begin{tikzpicture}[scale=0.9]
    % t = l/(2c)
    \begin{scope}[shift={(0,0)}]
        \draw[->] (-3.5,0) -- (3.5,0) node[right] {$x$};
        \draw[->] (0,0) -- (0,3.5);
        \draw[thick,red] (-3.75,0) -- (-1.25,1.5) -- (0,0);
        \draw[thick,red] (0,0) -- (1.25,1.5) -- (3.75,0);
        \node at (0,4) {$t = \frac{l}{2c}$};
        \node at (1.8,1.8) {$h/2$};
        \node at (-1.8,1.8) {$h/2$};
    \end{scope}

    % t = l/c
    \begin{scope}[shift={(8,0)}]
        \draw[->] (-5,0) -- (5,0) node[right] {$x$};
        \draw[->] (0,0) -- (0,3.5);
        \draw[thick,red] (-4.5,0) -- (-2.5,1.5) -- (-0.5,0);
        \draw[thick,red] (0.5,0) -- (2.5,1.5) -- (4.5,0);
        \node at (0,4) {$t = \frac{l}{c}$};
        \node at (3,1.8) {$h/2$};
        \node at (-3,1.8) {$h/2$};
        \draw[<->] (-0.5,-0.5) -- (0.5,-0.5) node[midway,below] {зазор};
    \end{scope}
\end{tikzpicture}
\end{center}

\subsection*{Ответ:}

Решение задачи Коши:
\begin{equation}
\boxed{u(x,t) = \frac{1}{2}\left[\varphi(x-ct) + \varphi(x+ct)\right]}
\end{equation}

где
$$\varphi(x) = \begin{cases}
h\left(1 - \frac{|x|}{l}\right), & |x| \leq l \\
0, & |x| > l
\end{cases}$$

Профиль струны представляет собой две треугольные полуволны амплитудой $h/2$ каждая, расходящиеся от центра в противоположных направлениях со скоростью $c$.

\end{document}
